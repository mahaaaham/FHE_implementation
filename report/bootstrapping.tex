\begin{section}{Mise en place d'un bootstrapping}
\begin{subsection}{Un point sur la sécurité}
JE N'AI PAS RETROUVÉ LE TERME UTILISÉ PAR MR. CASTAGNOS SUR LA SÉCURITÉ ICI.
CIRCULAIRE ? CYCLIQUE ? UN TRUC DE CE GENRE.\\
Nous avons vu que le système cryptographique que nous étudions est IND-CPA, ce qui est le niveau de sécurité théorique que l'on veut généralement. La première question qui se pose pour le bootstrapping est de savoir si l'on garde ce niveau de sécurité.\\
Le problème est que pour effectuer un déchiffrement homomorphe, il faut au moins un chiffré de la première clé secrète par la deuxième clé publique. \\
Il faut pouvoir s'assurer qu'un attaquant ne puisse pas en tirer d'information sur la première clé. Cependant, le système étant IND-CPA, cette propriété est toujours vérifié lorsque les deux clés sont indépendantes l'une de l'autre. \\
Il suffit donc de générer les clés indépendamment les unes des autres pour s'assurer d'avoir le niveau de sécurité désiré.
\end{subsection}
\begin{subsection}{Un premier découpage}
Afin de pouvoir effectuer un bootstrapping à partir de l'algorithme de déchiffrement \textbf{Dec}, nous allons avoir besoin de l'exprimer uniquement à partir d'opérations \textbf{NAND} sur des 0 et des 1. \\
Cela nous permettra de l'exécuter homomorphiquement. \\
Pour cela, nous considèrerons la décomposition binaire du secret et considèreront avoir un chiffré par la seconde clé de chaque bit du premier secret.
\end{subsection}
\begin{subsection}{Sommer des vecteurs avec le moins de NAND possible}
\end{subsection}
\begin{subsection}{Prendre la valeur absolue dans $\ZZq$}
\end{subsection}
\end{section}
