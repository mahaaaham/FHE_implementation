\begin{section}{Des librairies pour du FHE}
Plusieurs librairies open source implémentant divers FHE sont disponibles. 
On peut notamment en trouver une liste sur HomomorphicEncryption.org~\cite{homencrypt.org}
qui se décrit comme \og an open consortium of industry, government and academia to 
standardize homomorphic encryption \fg.

Nous proposons ici d'en évoquer deux:
\begin{itemize}
\item The Simple Encrypted Arithmetic Library (SEAL) \cite{seal}, dont nous avons tiré des paramètres  
\og réalistes \fg\footnote{Pas forcément pour nos machines et avec notre implémentation}
sécurisés et autorisant une profondeur de NAND non nulle (même si irréaliste: seulement 3, voir la 
sous-section~\ref{sec:choix_concrets} page~\pageref{sec:choix_concrets});
\item The Gate Bootstrapping API \cite{TFHE} qui implémente une variation du cryptosystème GSW;
\end{itemize}

\begin{subsection}{La librarie SEAL}
Acronyme de \og Simple Encrypted Arithmetic Library \fg, SEAL \cite{seal}
est une librairie écrite par le \og cryptography research group \fg \ de Microsoft en C++ sous 
licence MIT. Elle se propose d'implémenter deux FHE de seconde génération: 
BVS \cite{EPRINT:FanVer12} et CKKS \cite{AC:CKKS17}.

Son installation est facile\footnote{Sur Linux debian 4.9.0-8-amd64, nous avons dû
utiliser les backports debians pour avoir une version de cmake suffisament récente}
et un executable d'exemple est déjà présent pour tester diverses fonctionnalités de la librarie. De plus, la
documentation \cite{seal_manual_231}, malheureusement non à jour, indique quelques points théoriques autant du point de
vue mathématique que des choix de représentation des données.
\end{subsection}

\begin{subsection}{The Gate Bootstrapping API}
Notre présentation s'appuie sur celle donnée dans la page officielle (voir \cite{TFHE})
 qui est claire et bien documentée.

L'API Gate Bootstrapping est une librairie open source utilisable en C et C++ s'appuyant notamment sur des travaux de I. Chillotti, N. Gama, M. Georgieve et M. Izabachène 
(voir \cite{cryptoeprint:2016:870} et \cite{cryptoeprint:2017:430}). 

Elle utilise une version modifiée du cryptosysteme GSW (\cite{C:GenSahWat13})
étudié dans notre rapport, et permettant aussi bien du LHE que du FHE. C'est
pourquoi nous allons en parler plus en détails.

Ses performances sont interessantes; il est notamment indiqué dans la sous-section
4.2 de \cite{cryptoeprint:2016:870} que pour un ordinateur 64-bit simple coeur 
(i7-4930MX) cadencé à 3.00GHz, le bootstrapping se fait en un temps moyen de 52ms et 
que la clé de bootstrapping fait environ 24 Mo.

Pour arriver à de tels résultats, de nombreuses modifications et optimisations dans le codes ont été faites. Notamment,
le problème sur lequel s'appuie le cryptosystème n'est plus LWE mais une variante nommée TFHE, présentée dans
\cite{cryptoeprint:2016:870}. 

\begin{subsubsection}{Un exemple simple}
En plus d'une présentation de leur API, leur site contient un tutoriel sous forme de 3 fichiers de codes simples
permettant de simuler une communication chiffrée entre Alice et un cloud:
\begin{itemize}
\item Alice génère des clés, chiffre des données et les envoie accompagnées de la clé de bootstrapping au cloud;
\item Le cloud applique homomorphiquement une fonction, le minimum entre deux nombres, aux données et en renvoie le résultat à Alice;
\item Alice déchiffre le résultat.
\end{itemize}

\paragraph{}
Afin de manipuler la librairie, nous avons \og mis en forme \fg \ ces fichiers
en y ajoutant quelques modifications. Le tout est situé dans 
\path{using_tfhe_library} et il suffit de faire \path{make} 
pour compiler l'exécutable, à condition d'avoir la librairie tfhe installée.
	
Les fichiers sources ainsi que les headers contiennent normalement assez de
commentaire pour être lisibles. Voici un résumé du rôle de chaque fichier source:
\begin{itemize}
\item \path{alice.c} contient des fonctions permettant de générer les
clés ainsi que de chiffrer et de déchiffrer des messages;
\item \path{homomorphic_functions.c} contient deux exemples de fonctions 
applicables homomorphiquement: le minimum de deux nombres (déjà 
présent dans le tutoriel) et leur somme;
\item \path{cloud.c} contient une fonction permettant d'appliquer 
homomorphiquement une des fonctions de \path{homomorphic_functions.c}
sur des chiffrés puis d'enregistrer le résultat;
\item \path{example_communication.c} utilise les fichiers précédents
	pour simuler une communication entre Alice et le cloud.
\end{itemize}


	
\end{subsubsection} % un exemple simple fourni par leur tutorial
\end{subsection} % TFHE
\end{section}
