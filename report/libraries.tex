
\begin{section}{Des libraries C/C++ pour du FHE}
\begin{subsection}{The Simple Encrypted Arithmetic Library (SEAL)}
Acronyme de Simple Encrypted Arithmetic Library, SEAL est une librairie
écrite en C++ sous licence MIT. Elle propose :
\end{subsection}
\begin{subsection}{The Gate Bootstrapping API}
\end{subsection}
Notre présentation s'appuie sur celle donnée dans la page officielle (voir \cite{TFHE})
 qui est claire et bien documentée.

l'API Gate Bootstrapping est une librairie open source utilisable en C, C++ et 
s'appuyant notamment sur des travaux de I. Chillotti, N. Gama, M. Georgieve et M. Izabachène 
(voir \cite{cryptoeprint:2017:430} et  \cite{cryptoeprint:2016:870}). 

Elle utilise une version modifiée du cryptosysteme GSW (\cite{C:GenSahWat13})
étudié dans notre rapport, et permettant aussi bien du LHE que du FHE. 

Ses performances sont interessantes; il est notamment indiqué dans la sous-section
4.2 de \cite{cryptoeprint:2016:870} que pour un ordinateur 64-bit simple coeur 
(i7-4930MX) cadencé à 3.00GHz, le bootstrapping se fait en un temps moyen de 52ms.
de clée de bootstrapping d'environ 24MO.

Pour arriver à de tels résultats, de nombreuses modifications et optimisations dans le codes 
ont été faites. Notamment, le problème sur lequel s'appuie le cryptosystème n'est plus 
LWE mais TFHE, présenté dans les librairies suscitées.
\end{section}
