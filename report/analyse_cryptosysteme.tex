\begin{section}{Sécurité IND-CPA et paramètres pour un leveled GSW}
	\begin{subsection}{Sécurité du cryptosystème}

	\begin{definition}{Distance statistique}
	Soit $X$ et $Y$ deux variables aléatoires supportées par
	un ensemble $\mathcal{V}$ et à valeur 
	dans un groupe abélien $G$. On définit la distance 
	statistique entre $X$ et $Y$, notée $\SD(X,Y)$, 
	comme étant
	\[ \frac{1}{2} \sum_{v \in \mathcal{V}} |\mathbb{P}(X = v) -
	\mathbb{P}(Y = v)| \]

	\end{definition}

	\begin{definition}{Familles statistiquement indistinguables, calculatoirement indistinguables} \\
	On dira que deux familles ${\{X_i\}}_{i \in \NN}$, ${\{Y_i\}}_{i \in \NN}$
	de distributions sont 
	\begin{itemize}
	\item statistiquement indistinguables si la fonction $ i \longrightarrow \SD(X_i, Y_i) $ est négligeable.
	\item calculatoiremt indistinguables il n'existe pas d'automate polynomial probabiliste $\mathcal{A}$
	pouvant distinguer $X_i$ et $Y_i$ avec un avantage non négligeable.
	\end{itemize}
	\begin{prop}
	Si deux familles de distributions sont statistiquement
	indistinguables, elles sont calculatoirement indistinguables.
	\end{prop} 

	\end{definition}
	\begin{prop} \label{sd_add}
	Soit ${(X_i)}_{1\leqslant i \leqslant n}$ et resp .${(Y_i)}_{1\leqslant i\leqslant n}$
	deux n-uplets de distributions indépendantes.
	\[ \SD\left((X_1,\cdots,X_n), (Y_1, \cdots, Y_n)\right) \leqslant \sum_{i=1}^n \SD(X_i,Y_i) \]
	\end{prop}
	\begin{proof}
	Montrons le pour $n = 2$, la suite se déduisant par récurrence.
	\begin{align*}
		&\SD\left((X_1,X_2),(Y_1,Y_2)\right) \\
		&= \frac{1}{2} \sum_{(u,v)}\left| \PP(X_1=u)\PP(X_2=v) -
		\PP(Y_1=u)\PP(Y_2=v) \right| \\ 
		&\leqslant
		\frac{1}{2} \sum_{(u,v)}\left| \PP(X_1=u)\left(\PP(X_2=v) - \PP(Y_2 = v)\right)
		- (\PP(X_1 = u) - \PP(Y_1=u))\PP(Y_2=v) \right| \\
		&\leq
		\frac{1}{2} \sum_{(u,v)} \PP(X_1 = u)
		\left|\left(\PP(X_2=v) - \PP(Y_2 = v)\right) \right| + 
		\frac{1}{2} \sum_{(u,v)}\PP(Y_2=v)\left|(\PP(X_1 = u) - \PP(Y_1=u)) \right| \\
		&= \SD(X_1, Y_1) + \SD(X_2, Y_2)
	\end{align*}
	\end{proof}

	Le lemme suivant correspond au Claim 5.2 présent dans~\cite{STOC:Regev05}.

	\begin{lemme}
	Soit $G$ un groupe abélien fini. Pour $r > 1$ et 
	$\mathcal{F} \subset (g_1, \ldots, g_r) \in G^r$, on
	note
	 $s_\mathcal{F}$ la distribution aléatoire qui à un aléa
	 fait correspondre la somme $\sum_{i\in X} g_i$ pour un
	 sous-ensemble choisi de façon uniforme  $X\subset \llbracket
	 1, r \rrbracket$. 
	 D'autre part, on considère la distribution uniforme
	 $U$ sur $G$
	 Alors, on a: 
	 \[\mathbb{E}_{\mathcal{F}\subset G^r}(SD(s_\mathcal{F},U)) \leqslant
	   \sqrt{\frac{|G|}{2^r}}\]
	 Notamment, 
	 \[\mathbb{P}\left(SD(s_\mathcal{F},U) \geq
		 \sqrt[\leftroot{-3}\uproot{8}4\:]{\frac{|G|}{2^r}} \right) \leq
		 \sqrt[\leftroot{-3}\uproot{8}4\:]{\frac{|G|}{2^r}}
	 \]
	\end{lemme}
	\begin{proof}
		Remarquons que:
	\begin{align*}
		\sum_{h\in G} {\PP(s_F = h)}^2 &= 
		\PP\left(\sum_i b_i g_i = \sum_i b'_i g_i\right) \\
		& \leqslant \frac{1}{2^l} + \PP
		\left( \sum_i b_i g_i = \sum_i b'_i g_i |\:\: (b_i)_i \neq
		(b'_i)_i \right)
	\end{align*}
	Or, pour $(b_i)_i \neq (b'_i)_i$, 
		\[ \PP\left((g_i)_i\: :\: \sum_i b_i g_i = \sum_i b'_i
		g_i\right) = \frac{1}{|G|}\]
	D'où on déduit que:
	\[\mathbb{E}_\mathcal{F}\left(\sum_h {\PP(s_{\mathcal{F}} =
	h)}^2\right) \leqslant \frac{1}{2^l} + \frac{1}{|G|} \]

	Ce qui implique que:

\begin{align*}
\mathbb{E}_\mathcal{F}\left[ \sum_h\Big| \PP(s_F = h) - 1/|G|\Big| \right] 
&\leq
\mathbb{E}_\mathcal{F}
\left[ 
	{|G|}^{1/2}
	{ \left( \sum_h {\left(\PP(s_F = h) - 1/|G|\right)}^2
	\right)}^{1/2}
\right] 
\\
&= \sqrt{|G|} \:\:
\mathbb{E}_\mathcal{F}
	\left[ 
	{\left( \sum_h {\PP(s_F = h)}^2 - 1/|G|\right)}^{1/2} 
	\right] \\
&\leqslant \sqrt{|G|} \:\:
{\left( 
	\mathbb{E}_\mathcal{F}\left[ \sum_h  {\PP(s_F = h)}^2  \right]
- \frac{1}{|G|}
\right)}^{1/2} \\
	&\leqslant \sqrt{\frac{|G|}{2^l}}
\end{align*}

\end{proof}

	\begin{cor}
	Soit $G$ un groupe abélien fini. Pour $r > 1$ et 
	$\mathcal{F} \subset (g_1, \ldots, g_r) \in G^r$, on
	note
	 $s_\mathcal{F}$ la distribution aléatoire qui à un aléa
	 fait correspondre la somme $\sum_{i\in X} g_i$ pour un
	 sous-ensemble choisi de façon uniforme  $X\subset \llbracket
	 1, r \rrbracket$. 
	 Considérons alors le n-uplet $S_\mathcal{F} = (X_1, \ldots, X_r)$ où 
	 les $X_i$ sont indépendants de même loi
	 $s_\mathcal{F}$.
	 D'autre part, on considère la distribution uniforme
	 $U$ sur $G^r$
	 Alors, on a: 
	 \[\mathbb{E}_{\mathcal{F}\subset G^r}(SD(s_\mathcal{F},U)) \leqslant
	   \sqrt{r^2\frac{|G|}{2^r}}\]
	 Notamment, 
	 \[\mathbb{P}\left(SD(s_\mathcal{F},U) \geq
		 \sqrt[\leftroot{-3}\uproot{8}4\:]{r^2\frac{|G|}{2^r}} \right) \leq
		 \sqrt[\leftroot{-3}\uproot{8}4\:]{r^2\frac{|G|}{2^r}}
	 \]
	\end{cor}
	\begin{proof}
		Découle directement de la proposition précédente ainsi que de
		la proposition \ref{sd_add}
	\end{proof}

	\begin{prop}
	Supposons avoir pris des paramètres $(n, q, \chi, m)$ tels que l'hypothèse $\LWE$ soit vraie et soit $\tau > 0$.

	Si $m > (1+\tau)(n+1)\log(q)$, la distribution jointe $(A, RA)$, où 
	$A\in\ZZq^{m \times (n+1)}$ est une clé publique et $R \in{N \times m}$ est crée en tirant les coefficients
	uniformément dans $\{0,1\}$, est calculatoirement indistinguable de la
	distribution uniforme sur $\ZZq^{m \times (n+1)} \times \ZZq^{N
	\times (n+1)}$
	\end{prop}
	\begin{proof}
	On peut deja voir que comme $A$ est calculatoirement
	indistinguable de $U$, $(A, RA)$ l'est de $(U,RU)$ car on 
	peut facilement créer $(A, RA)$ (resp. $(U, RU)$) à partir de
	$A$ (resp. $U$).

	Il nous faut donc monter que $\mathcal{D}_1 = (U, RU)$ est calculatoirement
	indistinguable de $\mathcal{D}_2 = (U, V)$ où $V$ est uniforme.
	
	On peut alors utiliser le lemme précédent avec $G = \ZZq^{n+1}$
	et $r = m$ afin de voir qu'il existe une constante $\lambda > 0$
	telle que:
	\[\mathbb{E}_{\mathcal{U}\subset \ZZq^{m\times n+1}}(SD(RU,V)) \leqslant
		\sqrt{m^2\frac{q^{n+1}}{2^m}}\leqslant \lambda
	n\log(q)\sqrt{\frac{1}{q^{\tau(n+1)}}}=: f(n) \]
	Et, notant $Y = \{ U: \SD(RU,V) \geq \sqrt{f(n)}\}$, on obtient:
	\[\PP(U \in Y) \leqslant \sqrt{f(n)} \]
	où $f$ est négligeable en $n$. 

\vspace{0.2cm}
	Pour $(x,y)\in \ZZq^{m \times (n+1)} \times \ZZq^{m \times (n+1)}$, on a alors:
	\begin{align*}
	&\left|\PP(D_1 = (x,y)) - \PP(D_2 = (x,y))\right| \\ &\leqslant \PP(x\in Y)\:
	\Big|\PP(D_1 
	= (x,y)| x\in Y) - \PP(D_2 = (x,y)|x\in Y)\Big| + \PP(x \not\in Y)  \\
	&\leqslant |\PP(D_1 = (x,y) | x \in Y) - \PP(D_2 = (x,y)|x \in Y)| + \sqrt{f(n)} \\
	&\leqslant 2\sqrt{f(n)} 
	\end{align*}
	
	Ainsi, les deux familles de distributions sont statistiquement indistinguables, et donc calculatoirement
	indistinguables.	
	\end{proof}
	\begin{thm}{Sécurité IND-CPA}\\
	\label{ind_cpa}
	Sous les hypothèses de la proposition précédente, le
	cryptosystème est IND-CPA.
	\end{thm}
	\begin{proof}
	Comme un automate polynomial probabiliste ne peut pas distinguer
	$A_{s, \chi}$ de la distribution uniforme, on peut supposer que les coefficients 
	clé publique $A$ ont été tirés uniformément dans $\{ 0,1 \}$.

	\paragraph{}
	Considérons alors un chiffré 
	\[C = \flatten\left(\mu \cdot \id_N + \bitdecomp(R\cdot A)\right) \in
	\ZZq^{N\times N}\]

	On a:
	\[ \bitdecomp^{-1}(C) = \mu * \bitdecomp(\id_N) + R\cdot A\]

	Par la proposition précédente, un automate polynomial probabiliste $\mathcal{A}$
	ne peut pas distinguer $RA$ d'une matrice uniforme. On peut donc
	supposer que $RA$ est uniforme, et donc que le chiffrement est 
	un one-time pad.

	On en déduit qu'il n'existe pas d'automate polynomial probabiliste
	$\mathcal{A}$ permettant de déchiffrer efficacement les chiffrés de ce cryptosystème.
	\end{proof}
	
	\end{subsection}

Nous allons maintenant nous intéresser au choix de paramètres de notre
cryptosystème. Ils détermineront les degrés de sécurités et la profondeur des circuits calculables.

Sur ce point, deux approches sont possibles: une étude sur la sécurité asymptotique d'une famille 
de paramètres et une étude plus concrète, se demandant quelle est la sécurité d'un seul choix de paramètres. 
Nous jetterons ici un \oe{}il aux deux approches.

\paragraph{}
Posons ici notre contexte de travail
Nous nous intéresserons ici uniquement à l'algorithme de déchiffrement
\textbf{Decrypt} et ne considèrerons donc que des chiffrés de $0$ ou
$1$. Comme tout circuit booléen peut être construit uniquement avec des NAND, ce sera la seule opération
que nous considérerons. Notre façon de voir si un circuit peut-être évalué homomorphiquement puis déchiffré
avec succès sera alors de considérer sa profondeur en NAND.

\begin{subsection}{Choix asymptotique de paramètres pour un leveled GSW}
\label{sec:leveled}
\label{param_leveled}
	Nous allons ici que l'hypothèse~\ref{hyp_dwle} est vraie\footnote{Sachant qu'elle est faite dans un 
	article datant de 2017, il est possible qu'on sache aujourd'hui qu'elle est fausse. Il faudrait faire des
	recherches supplémentaires
	pour savoir ce qu'il en est}; autrement dit, que le problème DLWE est difficile avec les paramètres suivants:
\[ q \approx 2^{n^\epsilon}\quad \alpha q = n\quad \text{$m$ polynomial en $n$}\]
De plus, le théorème~\ref{ind_cpa} nécessite d'avoir 
\[m > (1+\tau)(n+1)\log(q) \]
pour un $\tau > 0$ pour que le cryptosystème soit IND-CPA.  Nous prenons alors: 
\[ m = 2(n+1)(\lfloor \log(q) \rfloor + 1) = 2(n+1) \bnorm{q} = 2N\]
Enfin, nous devons aussi respecter la condition de longueur pour pouvoir appliquer une profondeur de $L$ NAND à notre chiffré:
\[q > 8nm (1 + N)^L \]

En utilisant les inégalités $m < 2 (N+1)$ et $n < (N+1)$ dans cette dernière équation, on voit alors qu'il nous suffit d'avoir:

\begin{equation*}
q > 16 {(1+N)}^{L+2}
\end{equation*}

Ceci nous amène à un premier jeu de contraintes: 
\[ \begin{cases}\alpha  = n \cdot 2^{-n^\epsilon}  \\
	q = \lceil 2^{n^\epsilon}\rceil\\ 
	m = 2(n+1)\bnorm{q} \\  
	n^\epsilon > 4 + (L+2) \log\left( 1 + N\right)
	\end{cases} \]

Nous allons encore simplifier la dernière contrainte en utilisant
\begin{align*} (L+2) \log\left( 1 + N\right) &\leqslant (L+2) (2 + \log(N)) \\
&\leqslant (L+2) (2 + \log(n+1) +  \log(\bnorm{q})) \\
&\leqslant (L+2) (4 + \log(n) + \log(\log(q)) \\
&\leqslant (L+2) (5 + \log(n) + \log(n^\epsilon) \\
&\leqslant (L+2) (5 + (1+\epsilon)\log(n)) \\
&\leqslant 5(L+2) + (L+2)(1+\epsilon)\log(n)) \\
&\leqslant 3 L \log(n) \quad \text{pour $n$ assez grand}
\end{align*}

ce qui au final, nous donne:
\[ \begin{cases}
	\alpha  = n \cdot 2^{-n^\epsilon}=  \\
	q = \lceil 2^{n^\epsilon} \rceil\\ 
	m = 2(n+1)\bnorm{q} \\  
	n^\epsilon > 3 L \log(n)
	\end{cases}  \]
Il nous reste donc à trouver une valeur pour $n$, dépendant du paramètre de sécurité $\lambda$.  En posant $n = \rho^{1/\epsilon}$, on voit que la dernière contrainte devient:
\[ \frac{\rho}{\log(\rho)} > \frac{3 L}{\epsilon}  \]
et en prenant $\rho = \text{cst}\: a \log(a)$, cela devient:
\[\log(a)> \frac{3 L}{\epsilon\:\text{cst}\: a}\:(\log(a) + \log(\log(a)) + \log(cst))   \]
qui est vérifiée pour $a = L$ et $cst = 6/\epsilon$ car alors, on obtient:
\begin{align*}
& \log(L)> \frac{1}{\text{2}}\:(\log(L) + \log(\log(L)) + 1 + \log(3) - \log(\epsilon)) \\
&\Leftrightarrow  \log(L) - \log(\log(L)) - 1 - \log(3) > \log(\frac{1}{\epsilon})
\end{align*}
ce qui est vrai pour $L$ assez grand.  
On obtient donc 
\[n = \rho^{\frac{1}{\epsilon}}  = {\left(\frac{6 * L * \log(L)}{\epsilon}\right)}^{\frac{1}{\epsilon}}\]

\paragraph{}
On en déduit le théorème suivant:
\begin{thm}{leveled GSW}
	Pour $L$ et $\lambda$ assez grands, et sous l'hypothèse sur DWLE~(page \pageref{hyp_dlwe}), les paramètres suivants permettent de faire une profondeur de NAND de $L$
et rendent le cryptosystème GSW IND-CPA:
\[ \begin{cases} 
	n = \max\left(\lambda, \lceil 6/\epsilon \log(L) \log(\log(L))
	\rceil\right)  \\
	\alpha  = n \cdot 2^{-n^\epsilon}  \\
	q = \lceil 2^{n^\epsilon} \rceil\\ 
	m = 2(n+1)\bnorm{q} \\  
	\end{cases}  \]
\end{thm}

\begin{figure}[!ht]
\begin{tabular}{|l|c|c|c|}
\hline
& taille du secret & taille de la clé & taille d'un chiffré \\
\hline
$L = 10, \ \epsilon = 0.1$ & 131 octets & 264 Ko & 397 Ko \\
\hline
$L = 10, \ \epsilon = 0.5$ & 194 octets & 581 Ko & 3 Mo \\
\hline
$L = 100, \ \epsilon = 0.1$ & 410 octets & 3 Mo & 4 Mo \\
\hline
$L = 100, \ \epsilon = 0.5$ & 411 octets & 3 Mo & 4 Mo \\
\hline
$L = 1000, \ \epsilon = 0.1$ & 745 octets & 8 Mo & 13 Mo \\
\hline
$L = 1000, \ \epsilon = 0.5$ & 1 Ko & 15 Mo & 151 Mo \\
\hline
\end{tabular}
\caption{Taille des données suivant les paramètres étudiés précédemment avec $\lambda = 128$}
\label{size_boostrapping}
\end{figure}

Remarquons toutefois qu'il s'agit uniquement de paramètres \og théoriques \fg~ \ étant donné que la sécurité n'est
assurée asymptotiquement. Voyons maintenant si nous pouvons trouver des paramètres concrets permettant une utilisation 
raisonnable sur un ordinateur.

\end{subsection}

\begin{subsection}{Choix de paramètres concrets pour un leveled GSW}
\begin{subsubsection}{Présentation de lwe\_estimator}
\label{estimator}
	
Initialement utilisé dans l'article \cite{EPRINT:AlbPlaSco15}, \path{lwe_estimator} (disponible à l'adresse
\cite{estimator}) est un module de sagemath principalement maintenu par Martin Albrecht et destiné à estimer la résistance
de paramètres précis face à diverses attaques de paramètres précis pour le problème de learning with error.

\paragraph{}
Nous avons pensé qu'il pouvait être intéressant de l'utiliser afin de voir si nous pouvions trouver des paramètres
offrants une sécurité concrète, et non uniquement une famille de paramètres offrant une sécurité asymptotique.
Notons toutefois que la sécurité de notre cryptosystème n'est pas basée sur LWE mais DLWE, ce qui biaise dés le départ 
notre approche. De plus, la démonstration IND-CPA est elle aussi asymptotique et nous n'avons pas vu si il est possible 
de réduire une attaque du cryptosystème sur un seul jeu de paramètres à une attaque de DLWE.

Ces avertissements faits, regardons plus en détail comment fonctionne  \path{lwe_estimator}.


\paragraph{}
\textbf{estimate\_lwe :}

\paragraph{}
Pour estimer la résistance de paramètres choisis sur un panel d'attaques, on utilise la fonction \path{estimate_lwe}
dont une sortie typique est:

\flushleft
	
	\begin{lstlisting}
estimate_lwe(n, alpha=None, q=None, secret_distribution=True, m=oo,
             reduction_cost_model=reduction_default_cost,
             skip=("mitm", "arora-gb", "bkw"))
        \end{lstlisting}
	
\flushleft
Cette dernière prend en arguments les paramètres suivants qui lui permettent de créer une instance LWE
\begin{itemize}
\item Les paramètres $n, q$ habituels;
\item Un paramètre $\alpha$ égal à $s / q$ où $s$ est le paramètre de la gaussienne discrète utilisée comme paramètre
$\chi$ (à ne pas confondre avec le paramètre $\alpha$ de l'hypothèse \ref{hyp:proba});
\item d'autres arguments optionnels.
\end{itemize}
Le paramètre $m$ est optionnel car chaque attaque contre LWE évaluée utilise le nombre d'échantillons $m$ qui lui est nécessaire 
pour fonctionner et l'indique en sortie de la fonction. En réalité, même en indiquant un $m$ en option, la
sortie peut préciser des valeurs de $m$ supérieure à notre indication.

\paragraph{}
Elle retourne ensuite une résumé de la mémoire, du temps et d'autres paramètres spécifiques que nécessitent
diverses attaques contre LWE avec ce choix de paramètres . Le module contient 6 attaques différentes, mais n'en testera que
trois par défaut. Cela peut être modifié lorsque l'on appelle la fonction \path{estimate_lwe} via l'argument skip.
\flushleft
\begin{figure}
\begin{lstlisting}[mathescape=true]
sage: load("estimator.py")
sage: n = 2048; q = 2^60 - 2^14 + 1; $\alpha$ = 8/q; m = 2*n
sage: _ = estimate_lwe(n, $\alpha$, q, secret_distribution=(-1,1), 
	  reduction_cost_model=BKZ.sieve, m=m)
usvp: rop: =2^115.5,  red: =2^115.5,  $\delta_0$: 1.004975,  
      $\beta$:  288,  d: 4013,  m: 1964
 dec: rop: =2^127.1,  m:  =2^11.1,  red: =2^127.1,  $\delta_0$: 1.004663,  
      $\beta$: 318, d: 4237,  
      babai: =2^114.8,  babai_op: =2^129.9,  repeat: 7,  $\epsilon$: 0.500000
dual: rop: =2^118.4,  m:  =2^11.0,  red: =2^118.4,  $\delta_0$: 1.004864,  
      $\beta$: 298,  
      repeat:  =2^58.8,  d: 4090,  c:    3.909,  k: 30, postprocess: 13
\end{lstlisting}
\caption{Analyse de sécurité des paramètres tirés de la librairie SEAL}
\label{fig:seal_estimate}
\end{figure}


\vspace{0.2cm}
Une sortie de la fonction \path{estimate_lwe} est montrée dans la figure~\ref{fig:seal_estimate}.
Les principaux paramètres concernant le coût de chaque attaque sont \path{rop}, \path{mem} et \path{m}, où

\begin{itemize}
\item \path{rop} (ring operations) est une estimation du nombre d'opérations à effectuer afin de résoudre ce cas de LWE avec cette attaque.
\item  \path{mem} (memory) est une estimation de la mémoire qui sera exploitée.
\item \path{m} indique le nombre d'échantillons nécessaires pour résoudre le problème avec les paramètres choisis. 
A noter que l'on peut limiter le nombre d'échantillons disponibles pour l'attaquant
lorsqu'on appelle \path{estimate_lwe}.
\end{itemize}

\end{subsubsection}
\begin{subsubsection}{Proposition de choix sécurisés pour très faible profondeur de NAND} \label{sec:choix_concrets}
\paragraph{}

Comment par noter que \path{lwe_estimate} n'arrive pas à estimer la sécurité des paramètres de bootstrapping (étudiés
dans une section ultérieure), même 
avec un paramètre de sécurité $\lambda = 50$. Autre mauvaise nouvelle, la sécurité estimée des paramètres leveled
n'est pas bonne, ni pour $\epsilon = 1/3$, ni pour un epsilon bien plus petit, et ce même avec de grands $\lambda$, 
le paramètre \path{rop} reste proche de $2^{30}$ pour une des attaques. Cela peut signifier plusieurs choses:
\begin{itemize}
\item l'hypothèse de sécurité faite dans \cite{halevi} n'est plus réaliste;
\item le choix de $\epsilon$ doit être fait finement;
\item l'estimation est peut-être trop grossière.
\end{itemize}
Nous n'avons pas été dans le détail pour voir ce qu'il en est.

Nous nous sommes donc tournés vers d'autres choix de paramètres concrets, parmi ceux proposés dans 
\cite{estimator}.

\paragraph{}
En utilisant les paramètres suivants, tirés de l'API SEAL: 
\[n = 2048\quad q = 2^{60} - 2^{14} + 1 \quad \alpha = \frac{8}{q}\quad m = 2n \] 
On voit que l'estimation proposée par lwe indique  l'attaque la plus rapide demande $2^{115}$
opérations de base dans l'anneau $\ZZq$, soit un facteur de sécurité de 115. De plus, la condition de longueur
est respectée pour $L=3$ donc une profondeur de 3 NAND est possible. \\ 
\textbf{Taille:} secret: 15 Ko, une clé publique: 7.6 Mo, chiffré d'un message: 13 Go.

\paragraph{}
En utilisant les paramètres suivants, tirés de \cite{cryptoeprint:2015:755}:
\[n = 804\quad  q = 2^{31} - 19\quad \alpha = \frac{\sqrt{2\pi}*57}{q} \quad m = 4972\]
On voit que l'estimation proposée par lwe indique  l'attaque la plus rapide demande $2^{129}$ opérations de base dans
l'anneau $\ZZq$, soit un facteur de sécurité de 129. De plus, la condition de longueur
est respectée pour $L=1$ donc une profondeur de 3 NAND est possible. \\ 
\textbf{Taille:} secret: 3 Ko, clé publique: 5 Mo, chiffré d'un message: 2 Go.
\end{subsubsection}
\end{subsection}
\end{section}
