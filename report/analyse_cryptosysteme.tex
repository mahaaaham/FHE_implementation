\begin{section}{Analyse du cryptosystème: sécurité, profondeur des circuits}
	\begin{subsection}{Sécurité du cryptosystème}
	\begin{definition}{Distance statistique}
	Soit $X$ et $Y$ deux variables aléatoires supportée par
	un ensemble $\mathcal{V}$ et à valeur 
	dans un groupe abélien $G$. On définit la distance 
	statistique entre $X$ et $Y$, notée $\text{SD}(X,Y)$, 
	comme étant la somme:
	\[ \frac{1}{2} \sum_{v \in \mathcal{V}} |\mathbb{P}(X = v) -
	\mathbb{P}(Y = v)| \]

	\end{definition}
	\begin{prop}
	Soit $G$ un groupe abélien fini. Pour $r > 1$ et 
	$\mathcal{F} \subset (g_1, \ldots, g_r) \in G^r$, on
	note
	 $s_\mathcal{F}$ la distribution aléatoire qui à un aléa
	 fait correspondre la somme $\sum_{i\in X} g_i$ pour un
	 sous-ensemble choisi de façon uniforme  $X\subset \llbracket
	 1, r \rrbracket$. 
	 Considérons alors le n-uplet $S_\mathcal{F} = (X_1, \ldots, X_r)$ où 
	 les $X_i$ sont indépendants de même loi
	 $s_\mathcal{F}$.
	 D'autre part, on considère la distribution uniforme
	 $U$ sur $G^r$
	 Alors, on a: 
	 \[\mathbb{E}_{\mathcal{F}\subset G^r}(SD(s_\mathcal{F},U)) \leq 
	   \sqrt{r^2\frac{|G|}{2^r}}\]
	 Notamment, 
	 \[\mathbb{P}\left(SD(s_\mathcal{F},U) \geq
		 \sqrt[\leftroot{-3}\uproot{8}4\:]{r^2\frac{|G|}{2^r}} \right) \leq
		 \sqrt[\leftroot{-3}\uproot{8}4\:]{r^2\frac{|G|}{2^r}}
	 \]
	\end{prop}
	\begin{proof}
	\end{proof}
	\begin{prop}
		Supposons avoir pris des paramètres $(n, q, \chi, m)$
	tels que l'hypothèse $\LWE$ soit vraie. Alors pour $\epsilon>0$
	et $m > (1+\epsilon)(n+1)\log(q)$, la distribution jointe 
	$(A, RA)$ est calculatoirement indistinguable de la
	distribution uniforme sur $\ZZq^{m \times (n+1)} \times \ZZq^{N
	\times (n+1)}$
	\end{prop}
	\begin{proof}
		On voit qu'il suffit de montrer que la distribution $D_{ra} = (A, RA)$ 
		est indistinguable de de la distribution $D_{u} = (A, U)$ où $U$ est
		uniforme car, par hypothèse, la distribution $A$ ne peut être distinguée de 
		la distribution uniforme. 

		Supposons donc par absurde qu'il existe un automate $\mathcal{A}$
		poly-probabiliste distinguant ces deux distributions. On va
		alors créer un automate $\mathcal{B}$ distinguant  $A_{,\chi}$

		Notons $p_{u}(\mathcal{A})$ (resp. $p_{ra}$) la probabilité que $\mathcal{A}$
		estime qu'une entrée $(A,U)$ provienne bien de $D_{u}$ (resp.
		qu'une entrée $(A, RA)$ provienne bien de $D_{ra}$).

		Par hypothèse, il existe un entier $c > 0$ tel que:
		\begin{equation} |p_u - p_{ra}| > \frac{1}{n^c}\label{eq:sec1}\end{equation}

		Notons de plus, pour un secret $s$,  $p_{u}(s)$ la probabilité
		que $\mathcal{A}$ estime qu'une entrée $(A,U)$ créée avec le
		secret $s$ provienne bien de $D_{u}$ (resp.\ estime qu'une
		entrée $(A,RA)$ créée  avec le secret $s$ provienne bien de 
		$D_{ra}$).

		On constate que \[\EE_s(p_u(s)) = p_u(s), \quad \EE_s(p_{ra}(s))
				  = p_{ra}(s) \]
		Ainsi, on déduit de l'équation~\eqref{eq:sec1} qu'il existe 
		au moins une fraction $\frac{1}{2n^c}$ de secrets dans
		l'ensemble:
		\[ Y = \{ s : \:|p_u(s)) - p_{ra}(s)| \geq \frac{1}{2n^c}\} \]
	
		Comme $\frac{1}{2n^c}$ n'est pas négligeable, on voit qu'il
		suffit de créer un distingueur entre $D_{u}$ et $D_{ra}$ 
		restreints aux secrets $s\in Y$. En effet, si un tel
		distingueur $Z_{Y}$ existe, il suffit de l'utiliser pour 
		tout échantillon, même ceux crées à partir d'un $s \not\in Y$.
		Notant \og e = R \fg l'évènement signifiant
		que l'échantillon deviné provenait de la distribution $R$, on
		a:
		On aura alors:



	\end{proof}
	\begin{thm}
	Sous les hypothèses de la proposition précédente, le
	cryptosystème est IND-CPA.
	\end{thm}
	% la preuve de sécurité, sans trop évoquer les paramètres (le
	% minimum)
	\end{subsection}
	\begin{subsection}{Choix de paramètres, deux visions}
		% asymptotique vs concret
	\end{subsection}
	\begin{subsection}{Analyse asymptotique de la profondeur des circuits}
		% la preuve de sécurité
	\end{subsection}
	\begin{subsection}{Mise en place du Bootstrapping avec DEC}
	\end{subsection}
\end{section}
