\documentclass[15pt,usenames,dvipsnames]{beamer}
\usepackage[frenchb]{babel}
\usepackage[utf8]{inputenc}
\usepackage[T1]{fontenc}

% ----- rearranged from https://tex.stackexchange.com/questions/83882/how-to-highlight-
%                  python-syntax-in-latex-listings-lstinputlistings-command#83883 

% Default fixed font does not support bold face
\DeclareFixedFont{\ttb}{T1}{txtt}{bx}{n}{12} % for bold
\DeclareFixedFont{\ttm}{T1}{txtt}{m}{n}{12}  % for normal

% Custom colors
\usepackage{color}
\definecolor{deepblue}{rgb}{0,0,0.5}
\definecolor{deepred}{rgb}{0.6,0,0}
\definecolor{deepgreen}{rgb}{0,0.5,0}

\usepackage{listings}
\lstset{escapeinside={<@}{@>}}

%
% ------- PYTHON -------  
%

% Python style for highlighting
\newcommand\pythonstyle{\lstset{
language=Python,
xleftmargin=0cm,
numbersep = 0.5cm,
numbers=left, 
numberstyle=\tiny\bf, 
stepnumber=5, 
basicstyle=\ttm,
otherkeywords={self, echo, cat, mv, cp},             % Add keywords here
keywordstyle=\ttb\color{deepblue},
emph={MyClass,__init__},          % Custom highlighting
emphstyle=\ttb\color{deepred},    % Custom highlighting style
stringstyle=\color{deepgreen},
frame=tb,                         % Any extra options here (tb)
showstringspaces=false            % 
}}


% Python environment
\lstnewenvironment{python}[1][]
{
\pythonstyle
\lstset{#1}
}
{}

% Python for external files
\newcommand\pythonexternal[2][]{{
\pythonstyle
\lstinputlisting[#1]{#2}}}

% Python for inline
\newcommand\pythoninline[1]{{\pythonstyle\lstinline!#1!}}

%
% -------  C  -------  
%

% C style for highlighting
\newcommand\Cstyle{\lstset{
language=C,
xleftmargin=0cm,
numbersep = 0.5cm,
numbers=left, 
numberstyle=\tiny\bf, 
stepnumber=5, 
basicstyle=\ttm,
otherkeywords={sizeof},             % Add keywords here
keywordstyle=\ttb\color{deepblue},
emph={MyClass,__init__},          % Custom highlighting
emphstyle=\ttb\color{deepred},    % Custom highlighting style
stringstyle=\color{deepgreen},
frame=tb,                         % Any extra options here
showstringspaces=false            %
}}


% C environment
\lstnewenvironment{C}[1][]
{
\Cstyle
\lstset{#1}
}
{}

% C for external files
\newcommand\Cexternal[2][]{{
\Cstyle
\lstinputlisting[#1]{#2}}}

% C for inline
\newcommand\Cinline[1]{{\Cstyle\lstinline!#1!}}

%
% -------  ASSEMBLER -------  
%

% Assembler style for highlighting
\newcommand\assemblerstyle{\lstset{
language=[x86masm]Assembler,
xleftmargin=0cm,
numbersep = 0.5cm,
numbers=left, 
stepnumber=5, 
numberstyle=\tiny\bf, 
basicstyle=\ttm,
otherkeywords={sizeof},             % Add keywords here
keywordstyle=\ttb\color{deepblue},
emph={MyClass,__init__},          % Custom highlighting
emphstyle=\ttb\color{deepred},    % Custom highlighting style
stringstyle=\color{deepgreen},
frame=tb,                         % Any extra options here
showstringspaces=false            % 
}}


% assembler environment
\lstnewenvironment{Assembler}[1][]
{
\assemblerstyle
\lstset{#1}
}
{}

% assembler for external files
\newcommand\assemblerexternal[2][]{{
\assemblerstyle
\lstinputlisting[#1]{#2}}}

% assembler for inline
\newcommand\assemblerinline[1]{{\assemblerstyle\lstinline!#1!}}

%
% -------  BASH -------  
%

% Bash style for highlighting
\newcommand\bashstyle{\lstset{
language=Bash,
xleftmargin=0cm,
numbersep = 0.5cm,
numbers=left, 
numberstyle=\tiny\bf, 
stepnumber=5, 
basicstyle=\ttm,
otherkeywords={self, echo, cat, mv, cp},             % Add keywords here
keywordstyle=\ttb\color{deepblue},
emph={MyClass,__init__},          % Custom highlighting
emphstyle=\ttb\color{deepred},    % Custom highlighting style
stringstyle=\color{deepgreen},
frame=tb,                         % Any extra options here (tb)
showstringspaces=false            % 
}}


% Bash environment
\lstnewenvironment{bash}[1][]
{
\bashstyle
\lstset{#1}
}
{}

% Bash for external files
\newcommand\bashexternal[2][]{{
\bashstyle
\lstinputlisting[#1]{#2}}}

% Bash for inline
\newcommand\bashinline[1]{{\bashstyle\lstinline!#1!}}
 % where are all the options about syntax highlight.

% used defined commands
\newcommand{\LWE}{\text{LWE}_{n,q,\chi}}
\newcommand{\DLWE}{\text{DLWE}_{n,q,\chi}}
\newcommand{\SD}{\text{SD}}
\newcommand{\id}{\text{I}}
\newcommand{\flatten}{\text{Flatten}}
\newcommand{\bitdecomp}{\text{BitDecomp}}
\newcommand{\ZZ}{\mathbb{Z}}
\newcommand{\NN}{\mathbb{N}}
\newcommand{\RR}{\mathbb{R}}
\newcommand{\ZZq}{\mathbb{Z}_q}
\newcommand{\EE}{\mathbb{E}}
\newcommand{\PP}{\mathbb{P}}
\newcommand{\norm}[1]{{\|#1\|}_{\infty}}
\newcommand{\bnorm}[1]{{|#1|}_{\text{bin}}}

\newcommand{\pk}{\text{pk}}
\newcommand{\sk}{\text{sk}}
\newcommand{\evk}{\text{evk}}
\newcommand{\Dec}{\text{\textbf{Decrypt}}}
\newcommand{\Eval}{\text{\textbf{Eval}}}
\newcommand{\Keygen}{\text{\textbf{Keygen}}}
\newcommand{\Enc}{\text{\textbf{Encrypt}}}

\usepackage{ulem}
\usepackage{graphicx}
\usepackage{xcolor} % to put colors in equations, ex:  \color{red} ...  or
                    % \textcolor{red}{foo}: \
\usepackage[dvipsnames]{xcolor} % to have more colors.
%\usetheme{theme/beamerthememetropolis}
\usepackage{beamerthememetropolis}

% \usepackage{parallel}
\usepackage{tikz}
\usetikzlibrary{positioning}
\usetikzlibrary{calc}
\usetikzlibrary{decorations.pathreplacing}

\usepackage{xcolor} % to put colors in equations, ex:  \color{red} ...  or
% \textcolor{red}{foo}: \

\usepackage{wrapfig}

\usepackage{enumerate}

\title{Les FHE, c'est facheux.}
\date{\today}
\author{Lucas Roux \& Eric Sageloli}

\begin{document}

% ------------------------------------------------------------------------------------------------

\begin{frame}
  \maketitle
\end{frame}

%-------------------------------------------------------------------------------------------------

\begin{frame}
\frametitle{Ce que ne contient pas cette présentation}
\end{frame}

% ------------------------------------------------------------------------------------------------

\begin{frame}
\frametitle{Définition informelle d'un FHE}
\end{frame}

% ------------------------------------------------------------------------------------------------

\begin{frame} 
\frametitle{Un bref historique}
\end{frame} 

% ------------------------------------------------------------------------------------------------

\begin{frame} 
\frametitle{GSW, premier essai:}
\end{frame} 

% ------------------------------------------------------------------------------------------------
  
\begin{frame} 
\frametitle{GSW, second essai:}
\end{frame} 

% ------------------------------------------------------------------------------------------------
  
\begin{frame} 
\frametitle{GSW, version finale:}
\end{frame} 
  
% ------------------------------------------------------------------------------------------------
  
\begin{frame} 
\frametitle{GSW, version finale:}
\end{frame} 

% ------------------------------------------------------------------------------------------------

\begin{frame} 
\frametitle{GSW, version finale:}
\end{frame} 

% ------------------------------------------------------------------------------------------------

\begin{frame} 
\frametitle{Sécurité IND-CPA: le problème DLWE}
\end{frame} 

% ------------------------------------------------------------------------------------------------

\begin{frame} 
\frametitle{FHE avec bootstrapping}
\end{frame} 

% ------------------------------------------------------------------------------------------------

\begin{frame} 
\frametitle{Un premier découpage de Decrypt}
\end{frame} 

% ------------------------------------------------------------------------------------------------

\begin{frame} 
\frametitle{sommer deux listes: algorithme naif}
\end{frame} 
    
% ------------------------------------------------------------------------------------------------

\begin{frame} 
\frametitle{sommer deux listes: carry lookahead added}
\end{frame} 
% ------------------------------------------------------------------------------------------------

\begin{frame} 
\frametitle{Profondeur de NAND de l'algorithme de déchiffrement}
\end{frame} 

% ------------------------------------------------------------------------------------------------

\begin{frame} 
\frametitle{Paramètres avec bootstrapping obtenus, taille}
\end{frame} 

% ------------------------------------------------------------------------------------------------

\begin{frame} 
\frametitle{Paramètres avec bootstrapping obtenus, taille}
\end{frame} 

% ------------------------------------------------------------------------------------------------

\begin{frame} 
\frametitle{Des librairies pour faire du FHE}
\end{frame} 

% ------------------------------------------------------------------------------------------------

% \begin{frame}
%     \begin{column}{.48\textwidth}
%       \begin{figure}
%         \begin{center}
%           \begin{tikzpicture}[scale = 1, transform shape]
% 	    \node[inner sep=0pt] (alice) at (0,0)
    {\includegraphics[width=.25\textwidth]{../pictures/examples/alice.jpeg}};
\node[inner sep=0pt] (bob) at ($(alice) + (6,0)$)
    {\includegraphics[width=.25\textwidth]{../pictures/examples/bob.jpeg}};
\node[inner sep=0pt] (eve) at ($(alice) + (3,2)$)
    {\includegraphics[width=.25\textwidth]{../pictures/examples/eve.jpg}};
\node[inner sep=0pt] (alicec) at ($(alice) + (-1,0)$)
    {\includegraphics[width=.10\textwidth]{../pictures/examples/computer.jpg}};
\node[inner sep=0pt] (bobc) at ($(bob) + (1,0)$)
    {\includegraphics[width=.10\textwidth]{../pictures/examples/computer.jpg}};
\node  at ($(bob) + (1.2,-1)$)
    {\tiny déchiffrement\_AES.c};
\node at ($(alice) + (-1,-1)$)
{\tiny chiffrement\_AES.c};

\node[color = red] at ($(alice) + (-1,0.2)$) {M};
\node[color = red] at ($(bob) + (1,0.2)$) {\tiny $E_k(M)$};

\node[color = red] (C) at ($(alice) + (3,0.5)$) {\small$E_k(M)$};

\draw[->, thick] (alice) -- (bob);

%           \end{tikzpicture}
%         \end{center}
%       \end{figure}
%     \end{column}

%     \begin{column}{.48\textwidth}
%       \vspace{2cm}
%       \begin{itemize}
%       \item<2-> Et pourquoi pas?;
%       \item<3-> Ehehe
%         \begin{itemize}
%         \item $2^{128}$ entrées
%         \item 16 octets par sortie
%         \item $16 * 2^{128} = 2^{132}$ octets 
%         \end{itemize}
%       \end{itemize}
%     \end{column}
    
%   \end{columns}
% \end{frame}

% ------------------------------------------------------------------------------------------------

% \begin{frame}
%   \frametitle{Introduction : comment cacher la clé ?}
%   \begin{figure}
%     \begin{center}
%       \begin{tikzpicture}[scale = 0.6, transform shape]
%     {\includegraphics[width=.10\textwidth]{../pictures/examples/computer.jpg}};
%       \end{tikzpicture}
%     \end{center}
%   \end{figure}

%   Pour chaque table :
%   \begin{itemize}
%   \item $2^8$ entrées
%   \item 1 octet en sortie
%   \item taille : $2^8$ octets
%   \end{itemize}

%   taille totale : $16 * 2^8 = 2^{12}$ octet = 4 Ko

%   \pause
%   \vspace{0.7cm}

%   \begin{itemize}
%   \item La clé n'est plus complètement obfusquée.
%   \item<2->On aura besoin de $2032$ tables, stockés sur $500KB$
%   \end{itemize}
% \end{frame}

% ------------------------------------------------------------------------------------------------

% \begin{frame}
%   \frametitle{Introduction : contexte Whitebox}
% \begin{itemize}
%   \item Contexte classique:  blackbox
%   \item DRM 
%   \item nouveau contexte : whitebox
%   \end{itemize}
% \end{frame}

\end{document}
